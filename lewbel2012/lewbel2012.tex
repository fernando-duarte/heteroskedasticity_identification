\documentclass{article}
\usepackage{amsmath,amssymb,amsthm,geometry}
\geometry{margin=1in}

% ---------- shortcuts ----------
\newcommand{\E}{\mathbb{E}}
\newcommand{\cov}{\operatorname{Cov}}
\newcommand{\corr}{\operatorname{Corr}}
\newcommand{\var}{\operatorname{Var}}

% ---------- theorem environments ----------
\theoremstyle{plain}
\newtheorem{theorem}{Theorem}

\begin{document}

\section*{Heteroscedasticity‐Based Identification (Lewbel, 2012)}

\vspace{-0.8em}
\paragraph{Notation.}
Let $(Y_1,Y_2)'$ be endogenous, $X$ exogenous (contains a constant), and the $r\times 1$
vector $Z=g(X)$ any \emph{mean‑zero} function of~$X$ with $\E(ZZ')$ full rank.\footnote{%
Centring $Z$ rules out a constant in~$Z$ and guarantees
$\cov(Z,\varepsilon_j)=0$ while allowing $\cov(Z,\varepsilon_j^{2})\neq0$.
}
Define the \emph{linear‑projection residuals}
\[
W_j \;=\; Y_j-X'\E(XX')^{-1}\E(XY_j), \qquad
\varepsilon_j \;=\; Y_j-\beta_j'X-\gamma_jY_{3-j}, \quad j=1,2.
\]

% ---------- structural models ----------
\subsection*{1. Structural Forms}
\begin{align*}
\textbf{Measurement error:}\quad 
&Y_1 = X'\beta_1 + Y_2^*\gamma_1 + V_1, & Y_2 = Y_2^* + U,\\[2pt]
\textbf{Triangular:}\quad
&Y_1 = X'\beta_1+\gamma_1Y_2+\varepsilon_1, &
Y_2 = X'\beta_2+\varepsilon_2,\\[2pt]
\textbf{Simultaneous:}\quad
&Y_1 = X'\beta_1+\gamma_1Y_2+\varepsilon_1, &
Y_2 = X'\beta_2+\gamma_2Y_1+\varepsilon_2.
\end{align*}

\textbf{Note:} The measurement error model is rewritten as a triangular system with
$\varepsilon_1 = -\gamma_1 U + V_1$ and $\varepsilon_2 = U + V_2$, where $V_2$ is the 
residual from projecting $Y_2^*$ on $X$.

% ---------- core assumptions ----------
\subsection*{2. Core Assumptions}
\begin{enumerate}
\item[(A1)] $\E|ZZ'|<\infty$ and $\E(ZZ')$ nonsingular.
\item[(A2)] $\E(Z\varepsilon_j)=0$ and $\cov(Z,\varepsilon_1\varepsilon_2)=0$.
\item[(A3)] (Simultaneous case) the $r\times3$ matrix
      $\bigl[\cov(Z,W_1W_2)\;\;\cov(Z,W_1^{2})\;\;\cov(Z,W_2^{2})\bigr]$ has
      rank $2$.
\item[(A4)] $(\gamma_1,\gamma_2)$ and $(1/\gamma_2,1/\gamma_1)$ cannot both
      lie in the parameter space unless $\gamma_1\gamma_2=1$.
\end{enumerate}
\smallskip
\emph{Additional model‑specific conditions}
\begin{itemize}
\item Measurement error: $U$ is independent of $(X, Y_1, Y_2^*)$, $\E(XV_2) = 0$,
      $\cov(Z,V_2^{2})\neq0$.
\item Triangular: $\cov(Z,\varepsilon_2^{2})\neq0$.
\item Simultaneous: $\cov(Z,\varepsilon_j^{2})\neq0$ for at least one~$j$.
\end{itemize}

% ---------- internal instruments ----------
\subsection*{3. Internal Instruments (Two-Stage Least Squares)}
For the triangular system (including measurement error as a special case):
\[
\text{Instruments for } Y_2: \quad X \text{ and } (Z-\bar{Z})\hat{\varepsilon}_2
\]
where $\hat{\varepsilon}_2$ are residuals from regressing $Y_2$ on $X$.

% ---------- identification ----------
\subsection*{4. Point Identification}
\begin{theorem}[Triangular/Measurement Error] 
Under (A1)–(A2) and $\cov(Z,\varepsilon_2^{2})\neq0$,
$(\beta_1,\beta_2,\gamma_1)$ are point identified.
\end{theorem}

\begin{theorem}[Simultaneous] 
Under (A1)–(A4) and non‑degenerate $\cov(Z,\varepsilon_j^{2})$,
$(\beta_1,\beta_2,\gamma_1,\gamma_2)$ are point identified.
\end{theorem}

% ---------- GMM moments ----------
\subsection*{5. GMM Moment Conditions}

\paragraph{Parameter notation:}
\begin{itemize}
\item $\mu = \E[Z]$ is the population mean of $Z$ (unknown parameter)
\item $\bar{Z} = \frac{1}{n}\sum_{i=1}^n Z_i$ is the sample mean of $Z$ (observable)
\item In GMM: $\mu$ is estimated jointly with other parameters
\item In 2SLS: $(Z - \bar{Z})\hat{\varepsilon}_2$ is used directly as an instrument
\end{itemize}

\paragraph{(i) Triangular system (includes measurement error)}
\[
Q_{\text{TRI}}(\theta)=
  \begin{pmatrix}
    X\varepsilon_1\\ 
    X\varepsilon_2\\
    Z - \mu\\
    (Z-\mu)\varepsilon_1\varepsilon_2
  \end{pmatrix},
\quad
\E\bigl[Q_{\text{TRI}}(\theta)\bigr]=0,
\quad
\theta=(\beta_1,\beta_2,\gamma_1,\mu)'.
\]

\paragraph{(ii) Simultaneous system}
\[
Q_{\text{SIM}}(\theta)=
  \begin{pmatrix}
    X\varepsilon_1\\ 
    X\varepsilon_2\\
    Z - \mu\\
    (Z-\mu)\varepsilon_1\varepsilon_2
  \end{pmatrix},
\quad
\E\bigl[Q_{\text{SIM}}(\theta)\bigr]=0,
\quad
\theta=(\beta_1,\beta_2,\gamma_1,\gamma_2,\mu)'.
\]

\textbf{Special case:} If $\E[\varepsilon_1\varepsilon_2] = 0$, then $\mu$ can be dropped from $\theta$,
and the moment condition simplifies to $\E[Z\varepsilon_1\varepsilon_2] = 0$.

% ---------- set identification ----------
\section*{6. Set Identification (Relaxed A2)}

\paragraph{Relaxed moment.}
Allow
\[
|\corr(Z,\varepsilon_1\varepsilon_2)|
  \;\le\;
  \tau\,|\corr(Z,\varepsilon_2^{2})|,
  \qquad 0\le\tau<1.
\]

\begin{theorem}[Bounds for the triangular model]
\label{thm:bounds}
Assume conditions for the triangular model hold except that the above relaxed moment
replaces $\cov(Z,\varepsilon_1\varepsilon_2)=0$ and that
$\cov(Z,W_2^{2})\neq0$, $\var(W_2^{2})>0$.
Then $\gamma_1$ lies in the closed interval $\Gamma_1$ whose endpoints are the
(real) roots of
\begin{align}\label{eq:quad}
\frac{\bigl[\cov(Z,W_1W_2)\bigr]^{2}}{\bigl[\cov(Z,W_2^{2})\bigr]^{2}}
-\frac{\var(W_1W_2)}{\var(W_2^{2})}\tau^{2}
+2\Bigl(\frac{\cov(W_1W_2,W_2^{2})}{\var(W_2^{2})}\tau^{2}
      -\frac{\cov(Z,W_1W_2)}{\cov(Z,W_2^{2})}\Bigr)\gamma_1
+(1-\tau^{2})\gamma_1^{2}=0.
\end{align}
\end{theorem}

\paragraph{Properties.}
\begin{itemize}\itemsep0pt
\item $\tau=0\;\Rightarrow\;$ $\Gamma_1$ collapses to the
      point‑identified $\gamma_1$ of the triangular model.
\item $\Gamma_1$ widens monotonically in $\tau$.
\item For any $\gamma_1\in\Gamma_1$,
      $\beta_1=\E(XX')^{-1}\E\!\bigl[X(Y_1-Y_2\gamma_1)\bigr]$.
\end{itemize}

\paragraph{Estimation.}
Replace population moments in \eqref{eq:quad} by sample moments,
solve the quadratic for the two roots, and obtain standard errors by the
delta method.

\end{document}