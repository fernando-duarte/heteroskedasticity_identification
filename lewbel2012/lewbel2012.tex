\documentclass{article}
\usepackage{amsmath,amssymb,amsthm,geometry}
\usepackage{enumitem} % For custom list labels
\geometry{margin=1in}

% ---------- shortcuts ----------
\newcommand{\E}{\mathbb{E}}
\newcommand{\cov}{\operatorname{Cov}}
\newcommand{\corr}{\operatorname{Corr}}
\newcommand{\var}{\operatorname{Var}}
\newcommand{\plim}{\operatorname*{plim}}
\newcommand{\argmin}{\operatorname*{arg\,min}}

% ---------- theorem environments ----------
\theoremstyle{plain}
\newtheorem{theorem}{Theorem}
\newtheorem{remark}{Remark}

\begin{document}

\title{Heteroscedasticity-Based Identification}
\maketitle

\section{Preliminaries and Notation}

This document summarizes the heteroscedasticity-based identification strategy for models with endogenous regressors, as developed by Lewbel (2012). The method provides a way to construct valid instruments from the model's data when traditional external instruments are unavailable.

Throughout, \(Z=g(X)\) denotes a transformation of the exogenous variables, constructed to be mean-zero.\footnote{For simplicity, we write \(Z\) to mean the centered variable \(Z - \E[Z]\). In practice, this means we use the sample-demeaned version, \(Z_i - \bar Z\), in all calculations.}
All instruments are therefore understood to be mean-zero.

\section{Structural Forms and Reduced-Form Residuals}

Let the endogenous vector be \((Y_1,\,Y_2)'\), $X$ a vector of exogenous
variables (including a constant), and $Z=g(X)$ as above.

\begin{align*}
\textbf{Triangular:}\quad
& Y_1 = X'\beta_1 + \gamma_1 Y_2 + \varepsilon_1,
&\quad
  Y_2 = X'\beta_2 + \varepsilon_2, \\[3pt]
\textbf{Simultaneous:}\quad
& Y_1 = X'\beta_1 + \gamma_1 Y_2 + \varepsilon_1,
&\quad
  Y_2 = X'\beta_2 + \gamma_2 Y_1 + \varepsilon_2.
\end{align*}

Projecting each \(Y_j\) on \(X\) yields \emph{reduced-form residuals}
\(W_j\triangleq Y_j - X'(\E[XX'])^{-1}\E[XY_j]\).
A short calculation gives
\[
W_1 = \frac{\varepsilon_1+\gamma_1\varepsilon_2}{1-\gamma_1\gamma_2},\qquad
W_2 = \frac{\varepsilon_2+\gamma_2\varepsilon_1}{1-\gamma_1\gamma_2},
\]
with the triangular case obtained by setting~\(\gamma_2=0\).

\begin{remark}
The algebra follows directly from solving the two-equation system for the
reduced form and collecting the error terms.
\end{remark}

\section{Core Assumptions for Point Identification}

The method relies on the following key assumptions:
\begin{enumerate}[label=(A\arabic*)]
\item \textbf{Strict exogeneity.}
      $\E[\varepsilon_j\mid X]=0$ for $j=1,2$.  
      \emph{Time-series note:} in Section~\ref{sec:timeseries} we weaken this to a martingale-difference assumption and add HAC inference.

\item \textbf{Covariance restriction.}
      $\cov(Z,\varepsilon_1\varepsilon_2)=0$.  
      This holds automatically if the errors have a common factor structure
      \(\varepsilon_j = a_j u + \eta_j\) where the common factor \(u\) is uncorrelated with and mean independent of \(Z\).

\item \textbf{Instrument relevance via heteroscedasticity.}
      \begin{itemize}\itemsep2pt
      \item \emph{Triangular:}
            $\cov(Z,\varepsilon_2^2)\neq0$.
      \item \emph{Simultaneous:}
            the $r\times 2$ matrix
            $\Phi_W = [\,\cov(Z,W_1^2)\;\; \cov(Z,W_2^2)\,]$
            has rank~2.
      \end{itemize}

\item \textbf{Normalization (simultaneous case).}
      The parameter space for \((\gamma_1, \gamma_2)\) precludes the observationally
      equivalent pair \((1/\gamma_2,1/\gamma_1)\) unless
      \(\gamma_1\gamma_2=1\).
\end{enumerate}

\section{Triangular System: Closed-Form Identification and 2SLS}

\subsection{Closed-form}

\begin{equation}\label{eq:gamma1_id}
\gamma_1
  = \frac{\cov(Z,W_1W_2)}{\cov(Z, W_2^2)}.
\end{equation}

\begin{remark}[Why \eqref{eq:gamma1_id} identifies $\gamma_1$]
Under~(A2) the numerator simplifies to
\(\cov(Z,\varepsilon_1\varepsilon_2)+\gamma_1\cov(Z,\varepsilon_2^2)
    =\gamma_1\cov(Z,\varepsilon_2^2)\).
Because the denominator is non-zero by (A3),
\(\gamma_1\) is point identified.
\end{remark}

\subsection{Feasible two-step 2SLS}

\begin{enumerate}\itemsep2pt
\item \textbf{Generate residuals.}
      Regress \(Y_2\) on \(X\) via OLS and store the residuals
      $\hat\varepsilon_2 = Y_2 - X'\hat\beta_2^{\text{OLS}}$.

\item \textbf{Construct the heteroscedasticity-based instrument.}
      The generated instrument is \(IV = (Z-\bar Z)\hat\varepsilon_2\).

\item \textbf{First stage.} Regress the endogenous variable \(Y_2\) on the exogenous variables and the generated instrument, $[X, IV]$, to obtain fitted values \(\hat Y_2\).

\item \textbf{Second stage.} Regress \(Y_1\) on $[X,\hat Y_2]$
      to estimate \((\beta_1,\gamma_1)\).
\end{enumerate}

\paragraph{Practical guidance.}
Weak-instrument concerns apply because the instrument is generated.
Report the first-stage F-statistic on $IV$ and, where necessary,
use weak-IV-robust inference. Standard errors should account for the first-stage estimation step, typically via GMM or bootstrap.

\section{GMM Moment Conditions}

The identification strategy can be expressed more generally using GMM. The core idea is to form moment conditions that are zero at the true parameter values \(\theta\).

\subsection{Triangular system}
The parameter vector is $\theta = (\beta_1', \gamma_1, \beta_2')'$. The moment vector is formed by multiplying instruments by the model's structural errors, which are defined as \(\varepsilon_1(\theta) = Y_1 - X'\beta_1 - Y_2\gamma_1\) and \(\varepsilon_2(\theta) = Y_2 - X'\beta_2\).
\begin{equation}\label{eq:moment_tri}
Q_{\text{TRI}}(\theta) =
  \begin{pmatrix}
    X \cdot \varepsilon_1(\theta) \\[3pt]
    X \cdot \varepsilon_2(\theta) \\[3pt]
    Z \cdot \varepsilon_1(\theta) \cdot \varepsilon_2(\theta)
  \end{pmatrix}.
\end{equation}
The instruments are constructed directly from the exogenous data:
\begin{itemize}\itemsep0pt
    \item For the two mean equations, the instruments are the exogenous variables \(X\).
    \item For the covariance restriction, the instrument is the heteroscedasticity driver \(Z\).
\end{itemize}
Since the errors are functions of data and parameters, the entire moment vector is observable for any candidate \(\theta\).

\subsection{Simultaneous system}
The parameter vector is $\theta = (\beta_1', \gamma_1, \beta_2', \gamma_2)'$. The structural errors are now \(\varepsilon_1(\theta) = Y_1 - X'\beta_1 - Y_2\gamma_1\) and \(\varepsilon_2(\theta) = Y_2 - X'\beta_2 - Y_1\gamma_2\).
\begin{equation}\label{eq:moment_sim}
Q_{\text{SIM}}(\theta) =
  \begin{pmatrix}
    X \cdot \varepsilon_1(\theta) \\[3pt]
    X \cdot \varepsilon_2(\theta) \\[3pt]
    Z \cdot \varepsilon_1(\theta) \cdot \varepsilon_2(\theta)
  \end{pmatrix}.
\end{equation}
The instruments are again constructed from exogenous data: \(X\) for the mean equations and \(Z\) for the covariance restriction. Estimation proceeds by finding the parameter vector \(\hat\theta\) that minimizes the sample analog of the moment conditions, \(\bar{Q}(\theta)'W\bar{Q}(\theta)\), where \(W\) is a weighting matrix.

\section{Set Identification Under a Relaxed Covariance Restriction}

When assumption (A2) is weakened to allow for a small correlation,
\(|\corr(Z,\varepsilon_1\varepsilon_2)|
   \le \tau|\corr(Z,\varepsilon_2^2)|,\; \tau\in[0,1)\),
the parameter \(\gamma_1\) in the triangular model is set-identified rather than point-identified.

\begin{theorem}[Bounds for \(\gamma_1\) with \(\tau>0\)]
\label{thm:bounds}
$\gamma_1$ is contained in the closed interval whose endpoints are the
(real) roots of the quadratic equation in \(\gamma_1\):
\[
\frac{\cov(Z,W_1W_2)^2}{\cov(Z,W_2^2)^2}
-\frac{\var(W_1W_2)}{\var(W_2^2)}\tau^{2}
+2\!\left(
  \frac{\cov(W_1W_2,W_2^{2})}{\var(W_2^{2})}\tau^{2}
  -\frac{\cov(Z,W_1W_2)}{\cov(Z,W_2^{2})}
 \right)\!\gamma_1
 +(1-\tau^{2})\gamma_1^{2}=0.
\]
\end{theorem}
The interval collapses to the point estimate from \eqref{eq:gamma1_id} when \(\tau=0\) and widens as \(\tau \to 1\).

\paragraph{Identification of other parameters.}
The remaining parameters are identified conditional on a value of \(\gamma_1\) from its identified set.
\begin{itemize}\itemsep0pt
    \item The parameter \(\beta_2\) is always point-identified by OLS: \(\hat\beta_2 = (\E[X'X])^{-1}\E[X'Y_2]\).
    \item For each value \(\gamma_{1,k}\) in the identified interval for \(\gamma_1\), there is a corresponding identified value for \(\beta_1\), given by \(\beta_{1,k} = (\E[X'X])^{-1}\E[X'(Y_1 - Y_2\gamma_{1,k})]\).
\end{itemize}
The result is an identified set of parameter pairs \((\beta_1, \gamma_1)\) corresponding to the interval for \(\gamma_1\).

\section{Two Illustrative Extensions}

The core idea can be adapted to other contexts by creatively defining the heteroscedasticity-generating variable \(Z\).

\subsection{Conditional heteroscedasticity (Prono)}
\paragraph{Context and Model.} In a time-series setting, the structural model is a triangular system:
\begin{align*}
    Y_{1t} &= X_t'\beta_1 + \gamma_1 Y_{2t} + \varepsilon_{1t} \\
    Y_{2t} &= X_t'\beta_2 + \varepsilon_{2t}
\end{align*}
The key insight is that the error variance may be time-varying and predictable. Prono's extension assumes \(\varepsilon_{2t}\) follows a GARCH process, where its conditional variance is a function of past errors and variances:
\[
\var(\varepsilon_{2t}\mid\mathcal{F}_{t-1}) = \sigma_{2t}^2 = \omega + \alpha \varepsilon_{2,t-1}^2 + \beta \sigma_{2,t-1}^2.
\]

\paragraph{Procedure.} The fitted conditional variance from the GARCH model serves as the heteroscedasticity driver.
\begin{enumerate}\itemsep2pt
    \item Estimate the second equation by OLS to get residuals \(\hat\varepsilon_{2t} = Y_{2t} - X_t'\hat\beta_2\).
    \item Fit a GARCH(1,1) model to the residuals \(\hat\varepsilon_{2t}\) to obtain the series of fitted conditional variances, \(\hat\sigma_{2t}^2\).
    \item This fitted variance is the heteroscedasticity driver: set \(Z_t = \hat\sigma_{2t}^2\).
    \item Construct the generated instrument: \(IV_t = (Z_t - \bar{Z})\hat\varepsilon_{2t}\).
    \item Proceed with 2SLS as in Section 4.2, using \([X_t, IV_t]\) as instruments for \(Y_{2t}\) in the first structural equation. Use HAC-robust standard errors.
\end{enumerate}

\subsection{Regime heteroscedasticity (Rigobon)}
\paragraph{Context and Model.} The model is a simultaneous system where the error variances differ across observable, discrete regimes (e.g., pre- and post-policy change, or high- vs. low-volatility periods).
\begin{align*}
    Y_1 &= X'\beta_1 + \gamma_1 Y_2 + \varepsilon_1 \\
    Y_2 &= X'\beta_2 + \gamma_2 Y_1 + \varepsilon_2
\end{align*}
The key assumption is that for at least one error term \(\varepsilon_j\), its variance changes across regimes \(s\), while the covariance between the errors remains constant:
\begin{align*}
\var(\varepsilon_j \mid s) &\neq \var(\varepsilon_j \mid s') \quad \text{for } s \neq s' \\
\cov(\varepsilon_1, \varepsilon_2 \mid s) &= \text{constant for all } s.
\end{align*}

\paragraph{Procedure.} The regime indicators are used to generate the instrument.
\begin{enumerate}\itemsep2pt
    \item Estimate the second equation by OLS to get residuals \(\hat\varepsilon_{2}\).
    \item Create a set of dummy variables \(\{D_1, \dots, D_S\}\) for the regimes.
    \item The heteroscedasticity drivers are these dummies: set \(Z\) to be the vector of centered dummies, \(Z_s = D_s - p_s\), where \(p_s\) is the sample proportion of observations in regime \(s\).
    \item Construct the generated instrument(s): \(IV_s = Z_s \hat\varepsilon_{2}\).
    \item Proceed with 2SLS, using \([X, IV_1, \dots, IV_S]\) as instruments for \(Y_2\) in the main equation.
\end{enumerate}

\section{Time-Series Variant with Log-Linear Conditional Variances}
\label{sec:timeseries}

\paragraph{Context and Model.} This extension formalizes identification within a GMM framework for a stationary and mixing time series. The structural model is a simultaneous system:
\begin{align*}
Y_{1t} &= X_t'\beta_1 + \gamma_1 Y_{2t} + \varepsilon_{1t} \\
Y_{2t} &= X_t'\beta_2 + \gamma_2 Y_{1t} + \varepsilon_{2t}
\end{align*}
The key assumption is that the conditional variances are an explicit log-linear function of the exogenous variables \(X_t\):
\[
\log\sigma_{jt}^2 = X_t'\delta_j, \quad \text{where} \quad \varepsilon_{jt}\mid\mathcal{F}_{t-1}\sim(0,\sigma_{jt}^2) \quad \text{for } j=1,2.
\]
Since \((\delta_1,\delta_2)\) are assumed non-zero, \(X_t\) is correlated with \(\varepsilon_{jt}^2\), satisfying the instrument relevance condition (A3). The heteroscedasticity driver is defined as the centered exogenous variables, \(Z_t = X_t - \E[X_t]\).

\subsection{Moment vector}
The full set of parameters is \(\theta=(\beta_1',\beta_2',\gamma_1,\gamma_2,\delta_1',\delta_2')'\). The moment conditions implied by the model are \(\E[Q_t(\theta)]=0\), where:
\[
Q_t(\theta)=
\begin{pmatrix}
X_t\varepsilon_{1t}(\theta)\\[2pt]
X_t\varepsilon_{2t}(\theta)\\[2pt]
Z_t\varepsilon_{1t}(\theta)\varepsilon_{2t}(\theta)\\[2pt]
Z_t\!\left(\varepsilon_{1t}(\theta)^2-e^{X_t'\delta_1}\right)\\[2pt]
Z_t\!\left(\varepsilon_{2t}(\theta)^2-e^{X_t'\delta_2}\right)
\end{pmatrix}.
\]
\paragraph{Instruments as Functions of Observables.} The instruments are derived from the exogenous variables \(X_t\). The terms inside the expectation are functions of these instruments and the structural errors (e.g., \(\varepsilon_{1t}(\theta) = Y_{1t} - X_t'\beta_1 - Y_{2t}\gamma_1\)), which are themselves functions of the observable data and the parameter vector \(\theta\).
\begin{itemize}
    \item For the first two mean equations, the instruments are \(X_t\).
    \item For the error product and variance specification moments, the instrument is \(Z_t = X_t - \E[X_t]\), estimated in-sample as \(X_t - \bar{X}\).
\end{itemize}
Therefore, the sample average of \(Q_t(\theta)\) is a computable criterion function for any candidate parameter values.

\subsection{Estimation}
Estimation proceeds via a multi-step GMM procedure:
\begin{enumerate}\itemsep2pt
\item Obtain initial estimates of mean parameters via OLS/IV, and get residuals \(\hat\varepsilon_{jt}\).
\item Estimate the variance parameters \(\delta_j\) by regressing \(\log\hat\varepsilon_{jt}^2\) on \(X_t\).
\item Use the full set of moment conditions defined in \(Q_t(\theta)\) to form a GMM criterion function.
\item Minimize the GMM objective function using a HAC-robust weighting matrix (e.g., Newey-West) to obtain efficient, consistent, and asymptotically normal estimates of \(\hat\theta\). A conventional bandwidth choice is \(\ell_T=\lfloor4(T/100)^{2/9}\rfloor\).
\end{enumerate}

\section{Diagnostic Tests and Practical Checks}

\begin{itemize}\itemsep2pt
\item \textbf{Instrument relevance.}
      Report the first-stage \(F\)-statistic on the generated instrument(s); if \(F<10\), standard inference is unreliable, and weak-IV-robust tests should be used.
\item \textbf{Instrument validity.}
      If there are more heteroscedasticity drivers \(Z\) than needed for identification (i.e., the model is overidentified), a Hansen \(J\)-test of overidentifying restrictions can be used to test the validity of the moment conditions.
\item \textbf{Endogeneity of \(Y_2\).}
      The endogeneity of \(Y_2\) can be tested using a difference-in-Hansen test (C-statistic) or a Hausman-style test comparing the OLS and 2SLS estimates of \(\gamma_1\).
\item \textbf{Heteroscedasticity of \(\varepsilon_2\).}
      The crucial assumption (A3) can be checked with a Breusch–Pagan or White test for heteroscedasticity, by regressing the squared residuals \(\hat\varepsilon_2^2\) on the proposed driver(s)~\(Z\). A significant relationship provides evidence for instrument relevance.
\end{itemize}

\end{document}