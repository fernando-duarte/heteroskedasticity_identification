\documentclass{article}
\usepackage{amsmath,amssymb,amsthm,geometry}
\geometry{margin=1in}

% ---------- shortcuts ----------
\newcommand{\E}{\mathbb{E}}
\newcommand{\cov}{\operatorname{Cov}}
\newcommand{\corr}{\operatorname{Corr}}
\newcommand{\var}{\operatorname{Var}}
\newcommand{\plim}{\operatorname*{plim}}
\newcommand{\argmin}{\operatorname*{arg\,min}}

% ---------- theorem environments ----------
\theoremstyle{plain}
\newtheorem{theorem}{Theorem}
\newtheorem{remark}{Remark}

\begin{document}

\title{Heteroscedasticity‐Based Identification \\ (Lewbel, 2012 and Extensions)}
\maketitle

\section*{0 \quad Preliminaries and Notation}

Throughout the paper \(Z=g(X)\) denotes a \emph{centered} transformation of the
exogenous variables, i.e.\ \(\E[Z]=0\) by construction.\footnote{In practice we
demean $Z$ in sample; writing \(Z\) keeps formulas uncluttered.
The sample mean~$\bar Z$ appears only where an explicit finite‑sample
expression is needed (e.g.\ generated instruments).}
All instruments in what follows are therefore understood to be mean‑zero.

\vspace{-0.5em}
\paragraph{Shorthands.}
We pre‑define the operators \(\E\), \(\cov\), \(\corr\), and \(\var\)
and the probability limit \(\plim\) to streamline formulas.

% ---------- structural models ----------
\section*{1 \quad Structural Forms and Reduced‑Form Residuals}

Let the endogenous vector be \((Y_1,\,Y_2)'\), $X$ a vector of exogenous
variables (including a constant), and $Z=g(X)$ as above.

\begin{align*}
\textbf{Triangular:}\quad
& Y_1 = X'\beta_1 + \gamma_1 Y_2 + \varepsilon_1,
&\quad
  Y_2 = X'\beta_2 + \varepsilon_2, \\[3pt]
\textbf{Simultaneous:}\quad
& Y_1 = X'\beta_1 + \gamma_1 Y_2 + \varepsilon_1,
&\quad
  Y_2 = X'\beta_2 + \gamma_2 Y_1 + \varepsilon_2.
\end{align*}

Projecting each \(Y_j\) on \(X\) yields \emph{reduced‑form residuals}
\(W_j\triangleq Y_j - X'(\E[XX'])^{-1}\E[XY_j]\).
A short calculation gives
\[
W_1 = \frac{\varepsilon_1+\gamma_1\varepsilon_2}{1-\gamma_1\gamma_2},\qquad
W_2 = \frac{\varepsilon_2+\gamma_2\varepsilon_1}{1-\gamma_1\gamma_2},
\]
with the triangular case obtained by setting~\(\gamma_2=0\).

\begin{remark}
The algebra follows directly from solving the two‑equation system for the
reduced form and collecting the error terms; see Lewbel (2012, App.\ B) for
details.
\end{remark}

% ---------- core assumptions ----------
\section*{2 \quad Core Assumptions for Point Identification}

\begin{enumerate}
\item[(A1)] \textbf{Strict exogeneity.}
      $\E[\varepsilon_j\mid X]=0$ for $j=1,2$.  
      \emph{Time‑series note:} in Section~\ref{sec:TimeSeries} we weaken this to a martingale‑difference assumption and add HAC inference.

\item[(A2)] \textbf{Covariance restriction.}
      $\cov(Z,\varepsilon_1\varepsilon_2)=0$.  
      This holds automatically if the errors have a common factor
      \(u \perp X\); then \(\varepsilon_j = a_j u + \eta_j\) with
      \(u\) independent of \(Z\) implies the restriction.

\item[(A3)] \textbf{Instrument relevance via heteroscedasticity.}
      \begin{itemize}\itemsep2pt
      \item \emph{Triangular:}
            $\cov(Z,\varepsilon_2^2)\neq0$.
      \item \emph{Simultaneous:}
            the $r\times 2$ matrix
            $\Phi_W = [\,\cov(Z,W_1^2)\;\; \cov(Z,W_2^2)\,]$
            has rank 2; equivalently each column is linearly independent
            in~\(Z\).  See Lewbel (2012, Prop.\ 4).
      \end{itemize}

\item[(A4)] \textbf{Normalization (simultaneous case).}
      As in Lewbel (2012), the parameter space precludes the observationally
      equivalent pair \((1/\gamma_2,1/\gamma_1)\) unless
      \(\gamma_1\gamma_2=1\).
\end{enumerate}

% ---------- internal instruments ----------
\section*{3 \quad Triangular System: Closed‑Form Identification and 2SLS}

\subsection*{3.1 \; Closed‑form}

\begin{equation}\label{eq:gamma1_id}
\gamma_1
  = \frac{\cov(Z,W_1W_2)}{\cov(Z, W_2^2)}.
\end{equation}

\begin{remark}[Why \eqref{eq:gamma1_id} identifies $\gamma_1$]
Under~(A2) the numerator simplifies to
\(\cov(Z,\varepsilon_1\varepsilon_2)+\gamma_1\cov(Z,\varepsilon_2^2)
    =\gamma_1\cov(Z,\varepsilon_2^2)\).
Because the denominator is non‑zero by (A3),
\(\gamma_1\) cancels on the right‑hand side, yielding point identification.
\end{remark}

\subsection*{3.2 \; Feasible two‑step 2SLS}

\begin{enumerate}\itemsep2pt
\item \textbf{Generate residuals.}
      Regress \(Y_2\) on \(X\) via OLS and store
      $\hat\varepsilon_2 = Y_2 - X'\hat\beta_2^{\text{OLS}}$.

\item \textbf{Construct the heteroscedasticity‑based instrument.}
      \(IV = (Z-\bar Z)\hat\varepsilon_2\).

\item \textbf{First stage.} Regress \(Y_2\) on $[X,IV]$,
      obtain fitted values \(\hat Y_2\).

\item \textbf{Second stage.} Regress \(Y_1\) on $[X,\hat Y_2]$
      to estimate \((\beta_1,\gamma_1)\).
\end{enumerate}

\paragraph{Practical guidance.}
Weak‑instrument concerns apply because $IV$ is generated:
report the first‑stage F‑statistic on $IV$ and, where necessary,
use weak‑IV‑robust inference (Anderson–Rubin or Kleibergen–Paap
$r_k$ tests).  
Standard errors should be obtained from a two‑step GMM covariance
matrix or via bootstrap to account for first‑stage estimation.

% ---------- GMM moments ----------
\section*{4 \quad GMM Moment Conditions}

Let $\theta$ collect structural parameters and let
$\mu = \E[Z]=0$ by centering.  The identifying moments are

\paragraph{(i) Triangular system.}
\begin{equation}\label{eq:moment_tri}
Q_{\text{TRI}}(\theta) =
  \begin{pmatrix}
    X (Y_1 - X'\beta_1 - Y_2\gamma_1) \\[3pt]
    X (Y_2 - X'\beta_2) \\[3pt]
    (Z)(Y_1 - X'\beta_1 - Y_2\gamma_1)(Y_2 - X'\beta_2)
  \end{pmatrix}.
\end{equation}

\paragraph{(ii) Simultaneous system.}
\begin{equation}\label{eq:moment_sim}
Q_{\text{SIM}}(\theta) =
  \begin{pmatrix}
    X (Y_1 - X'\beta_1 - Y_2\gamma_1) \\[3pt]
    X (Y_2 - X'\beta_2 - Y_1\gamma_2) \\[3pt]
    (Z)(Y_1 - X'\beta_1 - Y_2\gamma_1)(Y_2 - X'\beta_2 - Y_1\gamma_2)
  \end{pmatrix}.
\end{equation}
Equations~\eqref{eq:moment_tri}–\eqref{eq:moment_sim} satisfy
$\E[Q(\theta)]=0$ under (A1)–(A3).

% ---------- set identification ----------
\section*{5 \quad Set Identification Under a Relaxed Covariance Restriction}

When assumption (A2) is weakened to
\(|\corr(Z,\varepsilon_1\varepsilon_2)|
   \le \tau|\corr(Z,\varepsilon_2^2)|,\; \tau\in[0,1)\),
$\gamma_1$ in the triangular model is set‑identified.

\begin{theorem}[Bounds for \(\gamma_1\) with \(\tau>0\)]
\label{thm:bounds}
$\gamma_1$ is contained in the closed interval whose endpoints are the
(real) roots of
\[
\frac{\cov(Z,W_1W_2)^2}{\cov(Z,W_2^2)^2}
-\frac{\var(W_1W_2)}{\var(W_2^2)}\tau^{2}
+2\!\left(
  \frac{\cov(W_1W_2,W_2^{2})}{\var(W_2^{2})}\tau^{2}
  -\frac{\cov(Z,W_1W_2)}{\cov(Z,W_2^{2})}
 \right)\!\gamma_1
 +(1-\tau^{2})\gamma_1^{2}=0.
\]
\end{theorem}

The interval collapses to a point when $\tau=0$
and widens monotonically as $\tau\!\to\!1$.

% ---------- extensions ----------
\section*{6 \quad Two Illustrative Extensions}

\subsection*{6.1 \; Conditional heteroscedasticity (Prono, 2013)}

Let $\var(\varepsilon_{2t}\mid\mathcal F_{t-1})=\sigma_{2t}^2$
follow a GARCH model; set $Z_t=\sigma_{2t}^2$.
Replacing population variances by the fitted
$\hat\sigma_{2t}^2$ yields the generated instrument
\((\hat Z_t-\bar Z)\hat\varepsilon_{2t}\),
and estimation proceeds exactly as in Section 3,
with HAC‑robust standard errors.

\subsection*{6.2 \; Regime heteroscedasticity (Rigobon, 2003)}

If exogenous regimes $s\in\{1,\dots,S\}$ shift
$\var(\varepsilon_j\mid s)$ but leave
$\cov(\varepsilon_1,\varepsilon_2\mid s)$ constant,
the centered regime dummies serve as $Z$ and satisfy (A2)–(A3),
delivering identification by the same 2SLS recipe.

% ---------- diagnostics ----------
\section*{7 \quad Diagnostic Tests and Practical Checks}

\begin{itemize}\itemsep2pt
\item \textbf{Instrument relevance.}
      Report the first‑stage \(F\)-statistic on generated instruments;
      if \(F<10\) adopt weak‑IV‑robust tests.
\item \textbf{Instrument validity.}
      Sargan/Hansen \(C\)-tests can be run by dropping the generated instrument
      and testing over‑identifying restrictions.
\item \textbf{Endogeneity of \(Y_2\).}
      Difference‑in‑Sargan (C‑statistic) compares the full and
      restricted instrument sets.
\item \textbf{Heteroscedasticity of \(\varepsilon_2\).}
      A Breusch–Pagan test of \(\hat\varepsilon_2^2\) on~\(Z\)
      provides evidence for assumption (A3).
\end{itemize}

% ---------- time‑series extension ----------
\section*{8 \quad Time‑Series Variant with Log‑Linear Conditional Variances}
\label{sec:TimeSeries}

Let $(Y_{1t},Y_{2t},X_t)$ be strictly stationary and
$\alpha$‑mixing with absolutely summable mixing coefficients.
Maintain the structural forms of Section 1 and set $Z_t=X_t-\mu=Z_t$.
Assume
\begin{equation}\label{eq:loglin_var}
\varepsilon_{jt}\mid\mathcal F_{t-1}\sim(0,\sigma_{jt}^2),\quad
\log\sigma_{jt}^2=X_t'\delta_j,\qquad j=1,2.
\end{equation}
Because $(\delta_1,\delta_2)\neq0$,
$\cov(Z_t,\varepsilon_{jt}^2)\neq0$, preserving relevance.

\vspace{-0.5em}
\subsection*{8.1 \; Moment vector}

\[
Q_t(\theta)=
\begin{pmatrix}
X_t\varepsilon_{1t}\\[2pt]
X_t\varepsilon_{2t}\\[2pt]
Z_t\varepsilon_{1t}\varepsilon_{2t}\\[2pt]
Z_t\!\left(\varepsilon_{1t}^2-e^{X_t'\delta_1}\right)\\[2pt]
Z_t\!\left(\varepsilon_{2t}^2-e^{X_t'\delta_2}\right)
\end{pmatrix},
\qquad
\E[Q_t(\theta)]=0,\;
\theta=(\beta_1,\beta_2,\gamma_1,\gamma_2,\delta_1,\delta_2)'.
\]

\subsection*{8.2 \; Estimation}

\begin{enumerate}\itemsep2pt
\item OLS/IV of the mean equations, obtain residuals $\hat\varepsilon_{jt}$.
\item Estimate \eqref{eq:loglin_var} by regressing $\log\hat\varepsilon_{jt}^2$
      on $X_t$; set $\hat\sigma_{jt}^2=e^{X_t'\hat\delta_j}$.
\item Instruments: $Z_t,\;Z_t\hat\varepsilon_{jt},\;
      Z_t(\hat\varepsilon_{jt}^2-\hat\sigma_{jt}^2)$ ($6$ per~$t$).
\item One‑step 2SLS or (if over‑identified) GMM with
      Newey–West weight matrix;
      a bandwidth such as
      $\ell_T=\lfloor4(T/100)^{2/9}\rfloor$ is conventional.
\end{enumerate}

Mixing ensures a serial‑correlation‑robust LLN/CLT, so
$\hat\theta$ is consistent and asymptotically normal.

% ---------- bibliography ----------
\bigskip
\begin{thebibliography}{9}\setlength{\itemsep}{0pt}

\bibitem{Lewbel2012}
Lewbel, A.\ (2012).
\newblock Using Heteroscedasticity to Identify and Estimate Mismeasured and
  Endogenous Regressor Models.
\newblock \emph{Journal of Business \& Economic Statistics}, 30(1), 67–80.

\bibitem{Prono2013}
Prono, T.\ (2013).
\newblock The Role of Conditional Heteroscedasticity in Identifying and
  Estimating Linear Simultaneous Equation Models.
\newblock \emph{Journal of Applied Econometrics}, 28(2), 338–356.

\bibitem{Rigobon2003}
Rigobon, R.\ (2003).
\newblock Identification Through Heteroskedasticity.
\newblock \emph{Review of Economics and Statistics}, 85(4), 777–792.

\end{thebibliography}

\end{document}
