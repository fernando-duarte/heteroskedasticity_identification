\documentclass[11pt]{article}
\usepackage{amsmath,amssymb}
\usepackage{booktabs}
\usepackage{geometry}
\geometry{margin=1in}

\title{Quick Reference: Lewbel Implementation Differences}
\date{\today}

\begin{document}
\maketitle

\section*{The Key Difference}

All three methods claim to implement Lewbel (2012), but they construct instruments differently:

\subsection*{Standard Lewbel (hetid \& Stata ivreg2h)}
\begin{enumerate}
\item First stage: $Y_2 = \beta_{20} + \beta_{21}X + \varepsilon_2$
\item Get residuals: $\hat{e}_2 = Y_2 - \hat{Y}_2$
\item Construct: $IV = (Z - \bar{Z}) \cdot \hat{e}_2$ where $Z = X^2 - E[X^2]$
\item Use instruments: $\{X, IV\}$ (just-identified)
\end{enumerate}

\subsection*{REndo's hetErrorsIV}
\begin{enumerate}
\item Reverse regression: $X = \alpha_0 + \alpha_1 Y_2 + u$
\item Get residuals: $\hat{u} = X - \hat{X}$
\item Construct: $IV = \hat{u} \cdot (Y_2 - \bar{Y}_2)$
\item Use instruments: $\{X, Y_2, IV\}$ (overidentified)
\end{enumerate}

\section*{Quick Verification Code}

\begin{verbatim}
# Generate data
library(hetid)
data <- generate_lewbel_data(1000, params)

# Standard Lewbel
e2 <- residuals(lm(Y2 ~ X, data = data))
iv_lewbel <- (data$Z - mean(data$Z)) * e2

# What REndo does (approximately)
u <- residuals(lm(X ~ Y2, data = data))
iv_rendo <- u * (data$Y2 - mean(data$Y2))

# Compare
cor(iv_lewbel, iv_rendo)  # High but not 1
\end{verbatim}

\section*{Results Summary}

\begin{table}[h]
\centering
\begin{tabular}{lcc}
\toprule
Method & Coefficient & Std. Error \\
\midrule
hetid/ivreg2h & -0.801 & 0.00094 \\
REndo (normal data) & -0.801 & 0.00100 \\
REndo (weak het.) & -0.800 & 0.00527 \\
\bottomrule
\end{tabular}
\end{table}

\textbf{Key Finding}: REndo uses a different identification strategy that can produce much weaker instruments when the heteroskedasticity pattern doesn't match its assumptions.

\end{document}